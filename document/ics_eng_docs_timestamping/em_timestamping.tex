\documentclass[11pt
  , a4paper
  , article
  , oneside
  %  , twoside
  , showtrims
 % , draft
]{memoir}

\usepackage{essdocs}
\usepackage[numbers]{natbib}
\usepackage[autostyle]{csquotes}

%% \usepackage{epstopdf}
%% \usepackage[pdf]{pstricks}
%% \usepackage{graphicx}

\setsecnumdepth{subsection}

\begin{document}
%\frontmatter
%% ESS Document Description
%%
\essdocdesc{Engineering Manual}

%% ESS Document Number
%%
\essdocnum{ESS-0085848}

%% Date
%%
\date{\today}

%% ESS Document Revision Number
%%
\essdocrev{1}

%% ESS Document State
%%
\essdocstate{Preliminary}

%% ESS Document Classification
%%
\essdocclass{ESS Use Only}

%% Document Title
%%
\title{ICS Engineering Manual}
\subtitle{for Time Stamping}
%% Document Author(s), if more than one author,
%% use \newline instead of \\ or \linebreak in order to seperate them
%% \author{Javier Cereijo Garcia \newline Jeong Han Lee }
\author{Javier Cereijo Garcia}

%% Document Reviewer(s) if more than one reviewer,
%% use \newline instead of \\ or \linebreak in order to seperate them
%\reviewer{Jeong Han Lee \newline Timo Korhonen (Chief Engineer)}
\reviewer{TBD}
%% Document Owner(s) if more than one owner,
%% use \newline instead of \\ or \linebreak in order to seperate them
\owner{ICS}

%% Document Approver(s) if more than one approver,
%% use \newline instead of \\ or \linebreak in order to seperate them
%\approver{Timo Korhonen (Chief Engineer)}
\approver{ICS}

\showtrimson

\esstitle
\newpage
\tableofcontents
\newpage

%\mainmatter


%%% Actual Document Start at below
\chapter{Overview}
At European Spallation Source (ESS), Integrated Control System (ICS) uses the Micro Research Finland (MRF) Timing System{\footnote{\url{http://www.mrf.fi/}}} as its timing system for the ESS site. It provides 3 main functionalities: event generation, data transmission and timestamping.

\section{Scope}
In this engineering manual it is explained how to configure an IOC to get its timestamps from the timing system. It includes the set up for the EVG, EVR and other EPICS records that need to get their timestamp from the timing system.\\

\textbf{Note that this is a very early draft document and should be updated as development progresses.}

\section{Target Audience}
This document is targeted to ICS engineers and technical stakeholders of the ESS timing system. It is assumed that the target audience has a technical background in the MRF Timing System, the EPICS development, and a Linux environment.


\chapter{System Description}
The timing system consists of an Event Generator (EVG) which converts timing events and signals to an optical signal distributed through fan-out units to an array of Event Receivers (EVRs). The EVRs decode the optical signal and produce hardware and software output signals based on the timing events. The EVG also distributes timestamping events that are used to transfer the precise real time to all EVRs.

The EVG gets the real time and a 1 pps (1 pulse per second) signal from an external clock, although for testing purposes it is also possible to generate the signals internally. The later should only be used for testing, since the clock is generated by software and is of limited accuracy, but it guarantees that all the timing modules (EVG and all connected EVRs) will have the same time and it comes closer to the production setup by allowing use of functions such as timestamping from device support.

The EVRs keep track of the clock by means of the dedicated timestamping events. In the mrfioc2 module EVR EPICS databases there is a record that gets synchronized with the EVG with the frequency of 1 Hz. In between that second, an internal EVR counter tracks progress of time, so that this records always holds the current time of the IOC, and it can be used as a timestamp source.

\clearpage

\chapter{System Environment}
Before describing the time stamping mechanism, it is mandatory to have proper system environment that consists of specific hardware and software lists. Here we will show the hardware and software lists, their block diagrams, and their setup in the ICS lab at ESS. The information shown in this chapter is used in the ICS Lab at ESS.

\section{Hardware}
Table~\ref{table:hwlist} shows the hardware list with the hardware used in the examples in this manual. It is possible to use different form factors than what is shown in the examples; for more information check the specific engineering manual.
\begin{table}[!htb]
  \centering
  \begin{tabular}{l|l|l}
    \toprule
    Hardware                 & Info                                         & Serial Number \\\midrule
    MRF cPCI-EVG-230         & \texttt{ICS TAG-805}                         & E283016       \\\midrule
    cPCI crate               & \texttt{ICS TAG-362}                         &               \\\midrule
    CPU                      & \texttt{ICS TAG-363}, hostname: icslab-ts01  & 43040574      \\\midrule
    MRF PCIe-EVR-300DC       & \texttt{ICS TAG-473}                         & M263057       \\\midrule
    MRF IFB-300              & \texttt{ICS TAG-352}                         & K472044       \\\midrule
    MRF Universal I/O module &                                              &               \\\midrule
    Adlink IPC               & \texttt{ICS TAG-837}, hostname: icslab-ipc01 & 93-41036-G10E \\\midrule
    Optical cables           & LC, Optical 850 nm                           &               \\\bottomrule
  \end{tabular}
  \caption[]{Hardware list.}
  \label{table:hwlist}
\end{table}

\section{Software}
Table~\ref{table:swlist} shows the software list and its environment.
\begin{table}[!htb]
  \centering
  \begin{tabular}{l|l}
    \toprule
    Item               & Version Info.                                                      \\\midrule
    CentOS Linux       & \texttt{7.4.1708}                                                  \\\midrule
    Kernel             & \texttt{3.10.0-693.17.1.el7.x86\_64}                               \\\midrule
    mrf kernel module  & version : \texttt{1} / srcversion \texttt{A998B22F1425D7388F5F7A7} \\\midrule
    E3                 & \texttt{2.5.4}                                                     \\\midrule
    EPICS Base         & \texttt{3.15.5}                                                    \\\midrule
    mrfioc2            & E3 module ver. \texttt{2.2.0}                                      \\\midrule
    devLib2            & E3 module ver. \texttt{2.9.0}                                      \\\bottomrule
  \end{tabular}
  \caption[]{Software list and its version information.}
  \label{table:swlist}
\end{table}


\clearpage

\chapter{Engineering Procedure}
This chapter provides the minimal information to configure an IOC to get its timestamps from the timing system. It is not intended to show a complete deployment of the timing system, but the basic parts that need to be included in the desired IOC start-up script to get timestamping.

\section{EVG configuration}
Listing~\ref{list:EVG.cmd} shows the basic start-up script for the EVG, with short comments for the important parts:
\begin{lstlisting}[
    style=termstyle,
    label={list:EVG.cmd},
    caption={Start-up script \texttt{startup\_EVG.cmd}. }
  ]
require mrfioc2, 2.2.0

epicsEnvSet("EPICS_CA_MAX_ARRAY_BYTES","10000000")

epicsEnvSet("IOC", "TIMESTAMP")
epicsEnvSet("DEV1", "EVG0")

epicsEnvSet("MainEvtCODE" "14")
epicsEnvSet("HeartBeatEvtCODE"   "122")
epicsEnvSet("ESSEvtClockRate"  "88.0525")

mrmEvgSetupPCI($(DEV1), "16:0e.0")
dbLoadRecords("cpci-evg230-ess.db",  "SYS=$(IOC), D=$(DEV1), EVG=$(DEV1), FEVT=$(ESSEvtClockRate), FRF=$(ESSEvtClockRate), FDIV=1")


iocInit()

# Configure a software thread to emulate 1PPS input
dbpf $(IOC)-$(DEV1):1ppsInp-Sel "Sys Clk"

# Configure the master event rate at 14 Hz
dbpf $(IOC)-$(DEV1):Mxc0-Prescaler-SP 6289464 # Divide event freq to get 14 Hz
dbpf $(IOC)-$(DEV1):TrigEvt0-EvtCode-SP $(MainEvtCODE)
dbpf $(IOC)-$(DEV1):TrigEvt0-TrigSrc-Sel "Mxc0"

# Wait 5 seconds and synchronize all timestamps
epicsThreadSleep 5
dbpf $(IOC)-$(DEV1):SyncTimestamp-Cmd 1

\end{lstlisting}

\section{EVR configuration}
For the timestamp to be available at the system with the EVR, the only requirement is to have the EVR initialized and receiving the transmission from the EVG. Listing~\ref{list:EVR.cmd} shows the basic start-up script for the EVR:
\begin{lstlisting}[
    style=termstylenumber,
    label={list:EVR.cmd},
    caption={Start-up script \texttt{startup\_EVR.cmd}. }
  ]
require mrfioc2,2.2.0

epicsEnvSet("EPICS_CA_MAX_ARRAY_BYTES","10000000")

epicsEnvSet("IOC", "TIMESTAMP")
epicsEnvSet("DEV1", "EVR0")

epicsEnvSet("MainEvtCODE" "14")
epicsEnvSet("HeartBeatEvtCODE"   "122")
epicsEnvSet("ESSEvtClockRate"  "88.0525")

mrmEvrSetupPCI("$(DEV1)",  "01:00.0")
dbLoadRecords("evr-pcie-300dc-ess.db","EVR=$(DEV1), SYS=$(IOC), D=$(DEV1), FEVT=$(ESSEvtClockRate)") (*@\label{evrdb}@*)

# Needed with software timestamp source w/o RT thread scheduling
var evrMrmTimeNSOverflowThreshold 100000


iocInit()

# Set delay compensation
dbpf $(IOC)-$(DEV1):DC-Tgt-SP 70
\end{lstlisting}

After being initialized the EVR will act as a time provider, as can be seen by checking \texttt{generalTimeReport} in the IOC shell, as shown in Listing~\ref{list:generaltimereport}:
\begin{lstlisting}[
    style=termstyle,
    label={list:generaltimereport},
    caption={\texttt{generalTimeReport} on the IOC shell with an EVR. }
  ]
6690f8a.icslab-ipc01.14335 > generalTimeReport
Backwards time errors prevented 1464 times.

Current Time Providers:
    "EVR", priority = 50
        Current Time is 2018-08-13 11:18:40.234652.
    "OS Clock", priority = 999
        Current Time is 2018-08-13 11:20:05.683007.

Event Time Providers:
    "EVR", priority = 50

\end{lstlisting}
This shows the 2 time providers for this system: the EVR and the last resort provider (OS clock). There is a fallback from higher priority providers (smaller value of priority, in this case the EVR) to lower priority providers (larger value of priority, usually the OS clock) if the higher priority ones fail.

In the case that the OS clock time is ahead of the EVR time, as shown in Listing~\ref{list:generaltimereport}, there will be some increasing \texttt{Backwards time errors prevention} notifications in \texttt{generalTimeReport}, and the timestamps might show a strange behaviour at the startup of the IOC, while the EVR time catches up with the OS clock time. For this reason it is recommended to have all systems set with OS clock times close to each other.

\subsection{Timestamping of records simultaneous with the timing system events}
Some records should have the same timestamp as the event that trigger them, for example to correlate their value to a specific beam pulse. In this case the \texttt{\$(IOC)-\$(DEV):Time-I} record is set to process on arrival of the desired event and to have exactly the same timestamp as that event. The value of the  \texttt{\$(IOC)-\$(DEV):Time-I} record is the current time expressed as a human-readable string \texttt{DATE HH:MM:SS}.

The easiest way of doing this is to modify line \ref{evrdb} in Listing~\ref{list:EVR.cmd} to add the macro \texttt{EVNT1HZ} equal to the event number that the \texttt{\$(IOC)-\$(DEV):Time-I} record should take its timestamp from, so that the line looks like:
\begin{lstlisting}[
    style=termstyle
  ]
dbLoadRecords("evr-pcie-300dc-ess.db","EVR=$(DEV1), SYS=$(IOC), D=$(DEV1), FEVT=$(ESSEvtClockRate), EVNT1HZ=$(MainEvtCODE)")
\end{lstlisting}
where the \texttt{EVNT1HZ} macro has the value of the \texttt{MainEvtCODE} variable, 14.

Alternatively this configuration can be made after the startup of the IOC with:
\begin{lstlisting}[
    style=termstyle
  ]
iocuser@icslab-ipc01: ~$ caput TIMESTAMP-EVR0:Time-I.EVNT 14
Old : TIMESTAMP-EVR0:Time-I.EVNT    125
New : TIMESTAMP-EVR0:Time-I.EVNT    14
iocuser@icslab-ipc01: ~$ caput TIMESTAMP-EVR0:Time-I.INP "@OBJ=EVR0, Code=14"
Old : TIMESTAMP-EVR0:Time-I.INP     @OBJ=EVR0, Code=125
New : TIMESTAMP-EVR0:Time-I.INP     @OBJ=EVR0, Code=14
\end{lstlisting}

After this set the .TSEL field of the record that should have the desired timestamp to point to the .TIME field of the \texttt{\$(IOC)-\$(DEV):Time-I} record of the EVR by adding this line after iocInit() in the startup script:
\begin{lstlisting}[style=termstyle]
dbpf examplerecord.TSEL TIMESTAMP-EVR0:Time-I.TIME
\end{lstlisting}



\clearpage

\backmatter
%%%\bibliographystyle{unsrt}
%%%\bibliographystyle{plainnat}
%%%\bibliographystyle{abbrvnat}
%\bibliographystyle{unsrtnat}
%%%\bibliographystyle{chicago}
%%%\bibliography{./ess_refs}
%\bibliography{timestamping}

\end{document}
